\documentclass{beamer}
\usepackage{beamerthemesplit}
\usepackage{graphicx,url}
\usepackage[brazil]{babel}
\usepackage[utf8]{inputenc}
\usepackage{multimedia}

\mode<presentation>
{
  \usetheme{Ilmenau}
  \setbeamercovered{transparent}
}

\newcommand{\eng}[1]{\textit{#1}}
\newcommand{\obra}{\textit{Em torno da romã}}
\newcommand{\goiaba}{\textit{Goiaba}}
\newcommand{\redmark}[1]{\textcolor{red}{#1}}
\newcommand{\graymark}[1]{\textcolor{gray}{#1}}
\newcommand{\tocar}{\textcolor{blue}{$\blacktriangleright$}}

\title{Produção do Genos: Pesquisa, extensão, ferramentas}
\author{Genos --- Grupo de pesquisa em teoria, composição e computação musical}
\date{22 de maio de 2009}

\logo{\includegraphics[scale=.15]{logo-genos}}

\begin{document}

\frame{\titlepage}

\section{O Genos}

\frame{
  \frametitle{Membros}
  \begin{itemize}
  \item Prof. Dr. Pedro Kröger (Pós doutorando)
  \item Marcos di Silva (Doutorando)
  \item Givaldo de Cidra (Graduando)
  \item Alexandre Passos (Mestrando em computação---Unicamp)
  \item Cristiano Figueiró (Doutorando)
  \item Guilherme Bertissolo (Doutorando)
  \item Prof. Dr. Amaro Borges (UFSM)
  \item Natanael de Souza Ourives (Graduando)
  \item Rodrigo Souto (Graduando em ciência da computação)
  \item Lucas Prado Melo (Graduando em ciência da computação)
  \end{itemize}
}

\frame{
  \frametitle{Realizações}
  \begin{itemize}
  \item 15 Cursos de extensão de 2007.2 a 2009.1:

    Csound, Pure Data, Composição Eletroacústica, Lilypond, Linux,
    Ferramentas Computacionais, \LaTeX, Áudio Básico, Recursos
    Musicais Utilizando Softwares Livres, Trilha sonora para cinema
    (em andamento)
  \item Pesquisas
    \begin{itemize}
    \item Metalinguagem para síntese sonora (XML e Lisp)
    \item Codificação para estruturas musicais
    \item Análise automática de harmonia (Rameau)
    \item Musicologia computacional
    \item Contornos para composição (Goiaba)
    \end{itemize}
  \end{itemize}
}

\frame{
  \frametitle{Apresentações em eventos}
  \begin{itemize}
  \item International Computer Music Conference --- ICMC 2008
  \item ANPPOM 2006 e 2008
  \end{itemize}
}

\frame{
  \frametitle{Mais ítens sobre produção}
  \begin{itemize}
  \item Projeto FAPESB 2007/2008 --- R\$ 29.000,00
  \item Pós-doutorado de Pedro Kröger, no CCRMA, da Stanford
    University (San Francisco, EUA)
  \item Apoio ao Encontro Nacional de Compositores Universitários ---
    Encun 2008
  \end{itemize}
}

\section{Pesquisa}

\subsection{Rameau}

\frame{
  \frametitle{Rameau: um sistema para análise automática de harmonia}
}

\subsection{Goiaba}

\frame{
  \frametitle{Contornos em Música}
  \begin{figure}
    \centering
    \includegraphics{5a-sinfonia}
  \end{figure}

  \href{run:audio/5a-sinfonia.ogg}{\tocar}

  \begin{figure}
    \centering
    \includegraphics[scale=1.4]{c-3120-simples}
  \end{figure}
}

\frame{
  \frametitle{Semelhança e possível coerência}
  \vspace{-3em}
  \begin{figure}
    \centering
    \includegraphics{ly-3120-5a-sinfonia}
    \hspace{1em}
    \includegraphics[scale=1.4]{c-3120-simples}
  \end{figure}

  \vspace{-5em}
  \href{run:audio/ly-3120-5a-sinfonia.ogg}{\tocar}
}

\frame{
  \frametitle{Expansão para outros elementos musicais}
  \begin{figure}
    \centering
    \includegraphics[scale=.9]{chord-densities-in-time}
    \hspace{1em}
    \includegraphics[scale=.9]{dynamics-in-time}
    \hspace{1em}
    \includegraphics[scale=1]{c-1023}
  \end{figure}
  \vspace{-2em}

  \href{run:audio/chord.ogg}{\tocar}
  \href{run:audio/dynamics.ogg}{\tocar}
}

\frame{
  \frametitle{Representações de contornos}
  \begin{itemize}
  \item Representação simbólica

    Contorno: Z(2 0 3 1)
  \item Representação gráfica
    \begin{figure}
      \centering
      \includegraphics[scale=1.5]{c-2031}
    \end{figure}
  \end{itemize}
}

\frame{
  \frametitle{Representação de operações}
  \begin{itemize}
  \item Retrógrado de X(1 2 3):\\
    $retr(X(1\;2\;3))=Y(3\;2\;1)$
  \item Transposição de X(1 2 3) com fator 2: $transp(X(1\;2\;3)\;2)=W(3\;4\;5)$
  \item Concatenação de operações:
    $transp(retr(inv(rot(X(1\;2\;3))\;2))\;3)=K(6\;4\;5)$
  \end{itemize}
}

\frame{
  \frametitle{Goiaba}

  Goiaba---Software para processamento de contornos musicais
}

\section{Ferramentas}

\frame{
  \frametitle{Linux: sistema operacional de alta produtividade}
}

\frame{
  \frametitle{Lilypond: produtividade editando partituras}
}

\frame{
  \frametitle{\LaTeX: produtividade com textos acadêmicos}
}

\subsection{Integração do grupo via web}

\frame{
  \frametitle{Ferramentas web}
  \begin{itemize}
  \item Wiki
  \item Repositório de arquivos
  \item Repositório de projetos (controle de versão)
  \item Sistema de gerenciamento de tarefas
  \end{itemize}
}

\subsection{Criação de estúdio com software livre}

\frame{
  \frametitle{Áudio e software livre}
}

\end{document}
